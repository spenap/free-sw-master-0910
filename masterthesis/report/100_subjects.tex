This chapter covers the teaching program taken during the master. First, the modules
covered will be stated, and the subjects in which they were divided will be presented.
After that, an in-depth review of each module will be provided.

The master is organized around several fields or modules:

\begin{itemize}
\item Introduction to Libre Software
\item Integration and Administration of Libre Software
\item Libre Software development on Web environments
\item Libre Software development on Destkop/Mobile environments
\item Studies of Libre Software communities
\end{itemize}

While those modules present all the topics to be covered in the master, in order
to make them easier to study and evaluate, they are separated into several different subjects:

\begin{description}

\item[Introduction to Libre Software]

During this subject, a general overview on Libre Software was introduced. Libre Software as a concept,
with its motivations and consequences, was presented and then its history and most important actors
were also contextualized.

It was also taught how Software Engineering techniques and Libre Software can benefit from each other,
and, for that, licenses, patents and legislation are studied.

\item[Systems Integration with Libre Software]

During this subject, a really wide range of technologies in the Systems Administration realm were presented.
From networking, understanding and setting up web servers, email servers and other communication mechanisms
to the security concerns most important when managing those systems, this subject made use of practices and
seminars to help acquiring a knowledge very valuable for a Systems administrator.

Besides, to better perform some of the common tasks found on the {\it sysadmin} day, Bash Scripting and Perl were also
used, so they could be more easily automated.

\item[Libre Software Development]

During this subject, the main differences between a proprietary project and a Libre one were presented. Tools used
to communicate and coordinate projects were introduced, as well as several development platforms,
programming languages. To achieve that goal, this subject made use of several practices and workshops, where
near-to-real-life projects were developed.

\item[Quality on Libre Software Development]

During this subject, the knowledge acquired in the previous one was improved by introducing other elements very
important for a successful project. As a result, documentation, packaging, versioning and distribution of Libre Software
were studied. The importance of notifying, detecting and solving errors was stated, and unit testing and continuous
integration were introduced.

%Calidad y certificaci\'on de software libre. Claves para el \'exito de un proyecto de software libre. Gesti\'on de grupos de voluntarios. Publicaci\'on y difusi\'on.

\item[Detailed Technical Studies of Libre Software Projects]

During this subject, some well known, existing Libre Software projects were studied, applying for that those techniques
and knowledge acquired in the previous subject. Among others, there were seminars on Android, GNOME, KDE, Open Solaris
and Telefonica I+D projects EzWeb and MyMobileWeb.

\item[Dynamics of Libre Software Communities]

During this subject, the following concepts were introduced
\begin{itemize}
\item Business Models based on Libre Software. Migrations to Libre Software.
Success stories and costs analysis
\item Methodologies, techniques and tools to analyze Libre Software Projects, communities and dynamics.
Data Sources and metrics used in Libre Software studies
\item Analysis of business view of Libre Software
\end{itemize}

\end{description}

\section{Introduction to Libre Software}

In the course of this subject we learnt about the most important people in this movement,
such as Richard Stallman, Eric Raymond, Linus Torvalds, Miguel de Icaza\dots
We also learnt about the most successful FLOSS-based business models\cite{flossmetrics:bm}
such as dual licensing, open core, product specialists\dots Licenses were
an important matter, too: we explored the most important ones, learnt the
differences between permissive and {\it copyleft} ones, and got to know the Open Source
Initiative and Free Software Foundation as the main entities to classify and
approve different Free and Open Source licenses. Later we analyzed, using the
Libresoft tools\cite{libresoft:tools, moodle:libresoft_tools, melquiades:libresoft_tools},
several important communities and projects.

To better consolidate the acquired information, we have to develop several works:
\begin{itemize}
\item About the business models part, I presented a business model focused on service-oriented
applications for mobile devices\cite{businessmodel}. While accounting numbers are not necessarily accurate,
all the process followed to obtain them, as well as sources and references, is perfectly valid
should I decide to become an entrepreneur. With regards to the ideas presented,
they are built on existing technologies and, in fact, some of them are being
exploited right now.

\item As an introductory work for evaluate Free and Open Source penetration in the
real world, I did a review of the then-current Operating Systems in the mobile devices world\cite{osmobile}.
While it was a valid review of the state of the art during Christmas, significant things
have changed in such a dynamic field.

\item For a practical training in the use of the Libresoft Tools, the Eye of GNOME project's
source code repository and mailing list were analyzed\cite{eogtraining}. That led to the identification of 
the most important contributors to the project.

\item Once that we had undergone that training, a bigger analysis was performed. For
that, I focused on the WebKit Project\cite{webkit:home, wiki:webkit}. My original idea was
quite ambitious, as I wanted to compare WebKit and Gecko, but soon I focused on WebKit.
While this work\cite{webkitreport} is really interesting, the conclusions I reached were quite biased: 
my main measure to evaluate code collaboration was the committer identifier,
while WebKit project stores the real author of a commit in the ChangeLog, which I ignored.
As a future extension of the work, a ChangeLog parser should be done. Wit that, and 
reusing the existing scripts, tools and procedures, a very useful report could be done.

\end{itemize}

\section{Systems Integration with Libre Software}

In the course of this subject we dealt with networking, systems administration, servers configuration
and management, version control with git, security\dots With the goal of achieving a good understanding
of the common necessities of a systems administrator, we also went through task automation, using Bash
scripting and Perl.

As we had previously done in the Introduction to Libre Software module, we also used some
practical work to consolidate the knowledge acquired:

\begin{itemize}
\item To better learn Bash scripting, we worked\footnote{\url{https://forge.morfeo-project.org/plugins/scmsvn/viewcvs.php/trunk/spenap/ias/scripting/?root=freeswmaster}} with regular expressions, making use
of tools like find, sed or grep. Besides regular scripts, we also worked with daemons,
learning about runlevels during the way.

\item About Perl development, we followed\footnote{\url{https://forge.morfeo-project.org/plugins/scmsvn/viewcvs.php/trunk/spenap/ias/perl/?root=freeswmaster}} the ``Learning by doing'' motto. In a brief and interesting tutorial,
we learnt the language basis and, in the last two exercises, we even played with Last.fm's API.

\item In the networking practices, we worked\footnote{\url{https://forge.morfeo-project.org/plugins/scmsvn/viewcvs.php/trunk/spenap/ias/networking/?root=freeswmaster}} with proxies, subnets, DNS, DHCP\dots Again, we were asked
for a good understanding of the inner workings, not only for results. 

\item Finally, as a logical complement of the all the previous process, we had to learn about security. Open
ports, running processes, different kinds of exploits, different types of securing a network... Due to the sensitive
nature of this matter, this work results are not publicly available. 
\end{itemize}

\section{Libre Software development on Destkop/Mobile environments}

During this subject we were trained in tools and technologies like those used in many Free and Open Source projects.
From the development point of view, we learn different languages, such as C + GObject and Python, and technologies, such as
GStreamer, libsoup, libxml, GTK+ and D-Bus, as well as git for source code versioning, Emacs as the IDE or gdb and valgrind for debugging and
memory profiling.
From the quality point of view, we were introduced to important concerns such as unit testing, internationalization,
localization, accessibility or even packaging.

Instead of individual practices like the ones we had previously done in other modules, this module made use of
workshops to consolidate the knowledge acquired. Initially, a project was chosen which would use several different technologies
--the said GStreamer, libsoup, libxml, GTK+ and D-Bus--. That way, JaMp was born.

JaMp is a Jamendo client: it has a backend which accesses Jamendo using libsoup, parses Jamendo's responses with
libxml and does multimedia playback using GStreamer. Those operations can be controlled from a frontend using GTK+, which
communicates with the backend by means of a D-Bus API. An iterative approach was applied to develop the project: starting with
the backend, written in C+GObject, then the D-Bus interface was created to communicate it with a PyGTK frontend, developed as the last part.
The development was done via patches sent to a mailing list to allow peer-review. Then, the module coordinators would check and integrate
those patches, pushing them to a \href{http://gitorious.org/mswl2010/jamp}{git repository at gitorious}. Later on, I achieved git committer status, also becoming one of the patch integrators.

Besides learning those new technologies and getting familiar with git, we also made extensive use of other good practices, such as
testing new additions or features introduced, using gdb for debugging or valgrind to check for memory leaks. We also got the project
infrastructure updated to support internationalization and localization (commits \href{http://gitorious.org/mswl2010/jamp/commit/cdb8a34f092f8011380437fb3d56f8e5b5c773a6}{cdb8a34} and \href{http://gitorious.org/mswl2010/jamp/commit/c26310539097d9a5bae4b8a1c9d1ced5ed0c8f65}{c263105}), and checked accessibility concerns in the User Interface.

While the initial target was the GNOME desktop, the project was meant to be adapted to
Maemo devices. After a Maemo workshop, the project build was updated
--requiring only minimal modifications as seen in commit \href{http://gitorious.org/mswl2010/jamp/commit/8e61de095265e5892bcac49f95a0def6029bbd00}{8e61de0}-- to target these devices, and a specific user
interface was started, using PyMaemo.

As a reward for the work done, we had the opportunity to present the work at the \href{http://2010.guadec.es/}{7th GUADEC-ES}. I was in charge of writing the application letter to present the work, and later I also did the
slides and \href{http://2010.guadec.es/guadec/programa}{gave the presentation}. Not only it was nice to share the work we had done, but
also to meet and learn what real hackers are doing.

\section{Libre Software development on Web environments}

In the course of the Web development module, different well-known technologies were learnt, such as PHP --with templates using Smarty--, CSS and Typo3, Java --later using Maven and the ZK framework--, Ruby on Rails, Django and JavaScript.

As we had previously done in the in the Desktop \& Mobile module, to consolidate the knowledge acquired, a Java project was chosen.
It was developed across several workshops, and made use of jUnit for unit testing, Maven for the project management and ZK
for the User Interface. The same principles of code collaboration were used, such as peer-review sending patches to a mailing list.
Besides, alternating those workshops there were different sessions to introduce Ruby on Rails and Django: for them, as well
as for the initial PHP sessions, specific tasks and practices were needed\footnote{\href{https://forge.morfeo-project.org/plugins/scmsvn/viewcvs.php/trunk/spenap/dew/django/?root=freeswmaster}{Django's Receitas Galegas} and \href{https://forge.morfeo-project.org/plugins/scmsvn/viewcvs.php/trunk/spenap/dew/ror/?root=freeswmaster}{Ruby on Rails' blog}}.

The project developed in this module was a Rich Internet Application for the Getting Things Done methodology\footnote{\url{http://git.igalia.com/cgi-bin/gitweb.cgi?p=riagtd.git;a=summary}}. Following the
Model-View-Controller pattern, first the model was created for the tasks, tasks lists, users and notes. All those pieces got unit tests
checking their functionality, and later a ZK User Interface was provided allowing users to create and manage their own lists of tasks.

\section{Studies of Libre Software communities}

During this module, which was held transversally to the others, several important Libre Software
communities were studied. Android, GNOME, KDE, Debian, OpenSolaris, Maemo and the Morfeo community, all had a session. During those sessions, we
were told how the community worked: from the way they took decisions to their infrastructures, conferences or funding. In some of them,
practical workshops were also held: that way we got a brief introduction to Qt on KDE, to Android application development in the Android
session, to Debian packaging on Debian's, and got to play with OpenSolaris or \href{http://ezweb.morfeo-project.org/}{EzWeb} and \href{http://mymobileweb.morfeo-project.org/}{MyMobileWeb} in the OpenSolaris and Morfeo sessions.

It is worth mentioning the session we had with Carlos Guerreiro, from Nokia. He gave us a great review of Maemo's history, pointed to some
interesting things he believes the future can bring in the mobile devices world and discussed with us his vision of {\it Free and Open Source Software}
from the corporate point of view.

%{\it eye of gnome, an\'alisis de webkit}
%{\it Android, GNOME, KDE, Solaris}
%{\it paqueteria, accesibilidad, internacionalizacion, pruebas de unidad}
%{\it escritorio, movil, web}
%{\it redes, seguridad, scripting, perl}
%{\it modelo de negocio, revision de sistemas operativos moviles}
