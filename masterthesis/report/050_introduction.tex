This report tries to cover, in a structured way, all the details of the work done
during the master, as well as the conclusions reached. In this first chapter, the
motivations which lead to the enrolment on the master will be discussed, the objectives
intended to be accomplished will be exposed, and the expectations set on this course
will be presented.

In the second chapter, the teaching work taken during the master will be introduced.
The master modules, which group the main areas of knowledge to be taught, will be
shown, followed by the actual subjects in which the modules were divided. Those subjects
will be described, showing both the theoretical and practical work done within them.

In the third chapter, the work done in Grilo at Igalia, as the master's
{\it practicum}, will be presented. Grilo project will be introduced, the tasks to
accomplish will be exposed, the results obtained will be shown, and the future
work and conclusions will be discussed.

In the fourth and last chapter, general conclusions for the whole master will be
presented. The work done in the different sections will be related and linked and the
initial objectives will be revisited, to check their completion status.

\section{Background and motivations}

I consciously know and use Libre Software\footnote{Instead of using the terms {\it Free Software},
{\it Open Source Software} or {\it Free and Open Source Software}, {\it Libre Software}
will be used mainly throughout this report} since my first year of studies at the Computer
Science Faculty in the A Coru\~na University. In the five-year period that followed,
I learnt about systems administration at a domestic level --local networking, firewalling,
scripting for tasks automation or even security up to some degree. Besides, the degree itself
made me know and use several programming languages, doing developments from small to medium-big.
Since the second year, my Operating System of choice was a GNU/Linux distribution: Mandrake at first,
Debian followed to be later replaced by Ubuntu. In the same way, I tried to use {\it Libre Software}
alternatives when possible, and also learn about Free Software philosophy, getting
to attend to one of R. M. Stallman's conference held in A Coru\~na.

While my knowledge about Libre Software was enough for my daily needs, I had the feeling
that I should somehow {\it formalize} and consolidate all that knowledge. The Free Software
Master provided me with that opportunity.

\section{Objectives}

In order to being able to better evaluate my progress in the master, I set myself a
number of objectives, distributed across several areas.

\begin{description}
\item[System Administration:] My knowledge of this field was limited to common, domestic management.
From setting up a local network to configuring a ssh server, I made use of tutorials or HOWTOs
from the Internet, but had not enjoyed a real training on the matter: I was sure I was {\it
reinventing the wheel} sometimes, or getting suboptimal solutions in many other occasions.
The objective would be to formalize as much knowledge as possible, replacing the {\it raw} one
with well formed one. At the same time, it would be important to get the ability to
acquire new knowledge in an optimal way.

\item[Mobile Development:] I have been interested in development for mobile devices for
some time, specially since I bought a Nokia N800. While I tried to start developing for
it many times, never succeeded. I wanted to acquire good practices for mobile development,
as well as some basic understanding of the most important APIs used in it. It would
be interesting not to limit that learning to Maemo-based devices, but also try to
target Android ones, or even mobile development in a generic way.

\item[Desktop Development:] While there are specific parts of mobile development which
make it very different from desktop one, there should be many points in common, from
sharing technologies to just making use of architectonic patterns, so that only the
User Interface would need to be changed. I wanted to know the best way to make a new
development so that making a port to the mobile world was plain easy. Related to that,
I was interested in getting the most from existing Desktops, such as GNOME and
KDE, understanding the balance between collaboration and competition. Which was the
role {\it freedesktop}\cite{freedesktop:home}\footnote{It was a relief to know that GNOME and KDE are
enjoying a great collaboration} was playing?

\item[Licensing:] How can several projects collaborate? When is it possible to
combine code coming under different licenses? Which are the main licenses {\it out there}?
I wanted to know about that: since my Computer Science's end of degree, I firmly
believe that reusing existing code is a key in the success of a development. But when
reusing, you have to take into account lots of things\dots

\item[Role played:] While I had some experience using and enjoying {\it Libre Software},
my role had always been a passive one: I had just benefited from {\it Libre Software},
never contributing to it. That was a really important objective of mine: get
to contribute to a {\it Libre Software} project, or at least, learn how to do it,
if it were the case.
\end{description}

\section{Expectations}

By looking at the master program, it was quite clear that all the topics I was interested
in were covered. However, I was aware that, in order to accomplish these objectives, I would 
need to work hard. The most obvious situation would be about contributing back to a
{\it Libre Software} project: while I could learn the mechanisms to do it, I would still
need to put time and effort to actually having something valuable to collaborate with.

The same could happen with desktop or mobile development: while the workshops would be interesting
to start covering topics, I knew I would need to use a lot of my own time to consolidate
that information.

My expectations were quite clear, however. I expected this master to put a lot of information
on the table, to point to a lot of resources, and to serve as a starting point to the {\it
Libre Software} world. I also had some expectations on start creating a professional network
of {\it Libre Software} developers and companies and, luckily, jump professionally into that
field. 
