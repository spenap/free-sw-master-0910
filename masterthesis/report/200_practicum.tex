In this chapter, the master's {\it practicum} work in the Grilo project at Igalia
will be described. First, a brief introduction about the project and the objectives set
by the Igalia staff will be provided, also including those requirements dictated
by the nature of the project.

After that, the actual work done on the project will be presented. Its
prerequisites --such as technologies used, other projects needed,
compilation issues\dots--, the difficulties found during the work --and their
solutions or workarounds, when possible--, and contributions originated in its scope
will be described.

Finally, conclusions and results for the work will be presented. Master's subjects
will be linked to the technologies, tools and projects used, and new possible improvements 
--and potential issues--  will be identified.

\section{Introduction}

Grilo\footnote{\url{http://live.gnome.org/Grilo}} is a framework focused on making media discovery and browsing easy for application developers. It provides
a high-level API that abstracts the differences between different providers, such as Youtube, Jamendo, Shoutcast,
Vimeo, etc. It is licensed under the LGPL and developed in C using GLib and GObject, and it is already a GNOME project.

As the practicum work for the master, collaboration was to be done in the following areas:
\begin{itemize}
\item Improve Grilo's binding support, by
\begin{itemize}
\item Studying the mechanisms for the automatic binding generation using GObject Introspection
\item Analyze Grilo's APIs and their conformity with the automatic binding generation infrastructure. Suggest improvements.
\item Perform the actual improvements on the introspection support
\item Test generated bindings
\end{itemize}
\item General improvements in the framework and plugins system, by
\begin{itemize}
\item Analyzing and fixing defects
\item Design and development of new features
\end{itemize}
\end{itemize}

The work being developed should be done as a real Free Software project. As a result,
the tools and methods employed had to be those common in this type of projects,
such as mailing lists, bugzilla, wiki, git\dots, being the communications always in English. The
preferred communication channel would be the mailing list, but the IRC would be used for real-time communications.

\section{Work done}

There were two main tasks for the {\it practicum}: on the one hand, there was the
binding support. On the other, more general improvements or fixes. However, many
of those general improvements were made during the course of the enhancement of
the bindings support.

\subsection{Getting started with Grilo}

In order to feel comfortable enough with Grilo to being able to collaborate to it,
I started reading its documentation. It provided a good general overview of the project,
detailed its API, and offered some code examples ready to be compiled.

Going through its documentation, I was able to contribute my first patches: fixing
some typos or mistakes found in the documentation, adding some configuration lines
to the examples so they could work and small thinks like that.

Thanks to those patches, I started becoming familiar with the code, its style and
organization, and the build process.

\subsection{Bindings support. What is GObject Introspection?}

Once that I felt I was ready to deal with Grilo, I started reading documentation
about binding generation, in general, and GObject Introspection, in particular.

It makes sense to build many kinds of applications using, at least, two different
levels and languages. On one side, there are those based on C + GObject: usually
they have great performances, being good for graphics, multimedia, and low level tasks
in general. On the other side, there are those with a managed, with
garbage collection, runtime, such as JavaScript, Python, Java, .NET\dots They are
better suited for application logic such as configuration, layout, dialogs and/or
tasks where performance is not that important.

It is important, then, to allow developers chose their language of choice. Not only
they will be more productive: they will be happier. While the GNOME platform has
been historically quite binding friendly, writing bindings usually involves an important
amount of job to be done by hand. At the same time, the amount of resources devoted
to a given project are limited, so having a developer occupy his time writing bindings
will reduce the amount of time he can spend adding new features or fixing bugs, for
example.

As a consequence, the GObject Introspection project was born\footnote{\url{http://live.gnome.org/GObjectIntrospection}}. With one of its major goals
being to become a convenient bridge between the low-level, high-performance world and
the high-level, ``less-efficient'' one, GObject Introspection solves the problem
of writing bindings by hand by putting all the needed {\it metadata} inside
the GObject library itself, using annotations in the comments. This will lead to less
duplicate work from binding authors, and a more reliable experience for binding consumers.
Additionally, because the introspection build process will occur inside the GObject libraries
themselves, a goal is to encourage GObject authors to consider shaping their APIs to be more
binding friendly from the start, rather than as an afterthought.

\subsection[How do automatic bindings work?]{How do automatic bindings with GObject Introspection work?}

In order to benefit from automatic bindings using GObject Introspection, a three step process must
be followed.
\begin{enumerate}
\item \textbf{In the library developer side:} First, a C + GObject library must be annotated
with some metadata --while there are some default assumptions, some annotations have to be made.

\item \textbf{In the GObject Introspection side:} The GObject Introspection scanner tool must
scan the library to extract the metadata and generate a {\it gir} file: an XML containing
type information, memory allocation and more. Then, the GObject Introspection compiler
will compile the {\it gir} file to generate a typelib.

\item \textbf{In the target language side:} The language wanting to use that introspection
information must provide a way to load that typelib.
\end{enumerate}

Since GObject Introspection is quite a young project, there are still few languages using it. JavaScript, from both
Seed and Gjs, or Python, via PyGObject, are two languages with a relatively mature support. Others, such as Lua, are
still being developed.

It is worth noting that GTK-Doc should also understand all the annotations used in
GObject Introspection --if not, it is a bug--, so they get recognized when scanning
a library.

\subsection{Bringing automatic bindings to Grilo}

Once that I knew how automatic bindings worked, I started to analyze which things
were missing in order to make Grilo automatically generate introspection bindings.

\subsubsection{Initial status}

Build support for GObject Introspection in Grilo had already been added by the time
I started contributing to it. There was also a small test case, written in JavaScript,
which made use of the automatic bindings. However, due to some previous changes in
the way Grilo's metadata keys were stored, the test did not work.

\subsubsection{The ParamSpec issue}

These keys had been previously just constants, defined as integers or strings. However,
since an important feature for Grilo was to allow developers define their own metadata
keys, an approach using GObject's ParamSpecs --aliased as Grl.KeyIDs--
had been introduced. While those ParamSpecs served their purpose just fine, 
their were totally {\it GObject Introspection unfriendly}. First it was needed
to annotate each of them as a ParamSpec, as the aliased type would not work at the
gir and typelib level. Then, it became clear that Gjs did not support them, and
probably would not in the short term.

The workaround chosen was to annotate those keys as {\it uints}: with that modification,
that simple JavaScript test worked fine. As the binding consumer is not required to know
that KeyIDs are ParamSpecs --he is even discouraged, since that is a specific implementation
detail-- and all the methods returning KeyIDs or receiving KeyIDs are ParamSpec-agnostic,
that workaround should work fine. However, it relies on being able to store a pointer
in an {\it uint}: in platforms when that assumption is wrong, it would fail.

\subsubsection{Going upstream}

After some discussion in the mailing list, I was suggested to use upstream versions
for GObject Introspection and Gjs: maybe in the latest versions the issues we were
experiencing would be solved. Until then, I had been using Ubuntu Lucid Lynx's packages,
which became quickly outdated when dealing with projects so young and fast-paced.
In order to start using the upstream versions, I set up JHBuild:

{\it JHBuild is a tool designed to ease building collections of source packages,
called ``modules''. JHBuild uses ``module set'' files to describe the modules available
to build. The ``module set'' files include dependency information that allows JHBuild
to discover what modules need to be built and in what order.}

By using JHBuild with the GNOME 3 ``module set'', I started using GObject Introspection and
Gjs from upstream. However, since that did not fix the ParamSpec issue, I decided to test
the bindings from another language: Python.

\subsubsection{Using Grilo from Python: PyGObject}

Getting JHBuild to build latest PyGObject was a bit tricky: the documentation was
outdated, and pointed to a no longer valid PyGI module. With Javier Jard\'on and Nacho's
(Ignacio Casal) help, I realized that only PyGObject was necessary.

As PyGObject did support ParamSpecs, it became the {\it official} language for testing
the bindings. I then started to annotate all the methods in Grilo, one by one. In order
to be able to check the annotations and functionality of the bindings, I cloned
an existing test application --grilo-test-ui-- from C to Python, using PyGObject.

\subsubsection{Unit testing from Python}

Although writing a PyGObject clone for grilo-test-ui proved very useful, to identify
existing issues in some of the methods more commonly used, there was a lot of code
unchecked. To achieve a better coverage, I suggested the creation of unit tests in
Python, using PyGObject to invoke Grilo's methods.

These unit tests had a double effect: not only they were useful to identify parts
of the code not introspectable or presenting wrong annotations --which can lead
to wrong {\it free}s and / or crashes-- but also to test the library's behavior.
Right now, Python unit tests are fully integrated in the project's build --provided
that GObject Introspection support is found in the target system--, covering Grilo's
most common methods, and having led to the identification of several bugs.

\subsubsection{Current status of the bindings}

Right now, save for three functions non-introspectable, all the API can be accessed
via GObject Introspection from Python. In Gjs, as ParamSpecs are still unsupported, only the
API which does not use them works. However, when the ParamSpecs are annotated as {\it uints},
Gjs can achieve the same degree of functionality as Python.

Currently, a Gjs' branch has been suggested, containing a clone of the C application grl-inspect,
which provides an overview of the currently available plugins, and allows checking JavaScript's
support of the bindings in a more complex way than the initial test case.
While that branch has not been pushed to the git repository yet, it is available\footnote{\url{http://mail.gnome.org/archives/grilo-list/2010-October/msg00011.html}} in
the mailing list archives.

\subsection{Other improvements and fixes}

Besides the work done on the annotation field, several other improvements were made, such as
code rewritings, build fixes or improvements or documentation enhancements. Among them,
it is worth mentioning Vala's metadata improvements, needed for generating its binding, or
the rewriting of the {\it get\_sources} method, so it could return a GList instead of an array,
not supported in Gjs.

The full list of contributions to the Grilo core is as follows:
\begin{verbatim}
Simon Pena (64):
      doc: Added grl_init call
      doc: Minor corrections
      doc: Corrected obsolete tags
      core: Replaced grl-config defines with functions
      core: Replaced grl-media defines with functions
      core: Replaced grl-media-video defines with functions
      core: Replaced grl-media-image defines with functions
      core: Replaced grl-media-audio defines with functions
      annotations: Annotated grilo.c
      annotations: Annotated grl-plugin-registry
      annotations: Annotated grl-media-source
      annotations: Annotated grl-media-source callbacks
      core: Make grl-metadata-key more introspection friendly
      annotations: Annotated grl-multiple
      annotations: Annotated grl-data
      python: Cloned grilo-test-ui
      core: remove uneed frees
      core: removed wrong free at GrlMediaPlugin finalize()
      annotations: Fixed 'lookup_source' annotation
      core: Remove value destroy_func from the plugins' hash table
      core: check for 'key-depends' implementation
      annotations: Added missing colon
      annotations: Replaced GObject.ParamSpec* with GObject.ParamSpec
      annotations: Replaced Grl.Media* with Grl.Media
      annotations: Replaced Grl.MediaSource* with Grl.MediaSource
      annotations: Skipped non introspectable functions
      core: Made GrlPluginInfo introspectable
      annotations: Added missing transfer modes
      annotations: Annotated grl-metadata-source
      annotations: Annotated metadata_source cb GErrors as uints
      annotations: Fixed 'get_info_keys' annotation
      core: Replaced grl-media-plugin defines with functions
      annotations: Annotated 'register_metadata_key'
      annotations: Removed unnecessary annotations
      annotations: Media in cb is transfer full
      annotations: Fixed grl_data_get_keys transfer mode
      doc: Updated examples to use new log system
      core: Make get_sources return a GList
      core: Make get_sources_by_operations return a GList
      annotations: Fixed grl_init annotations
      doc: Improve GrlMediaAudio documentation
      doc: Improve GrlMediaImage documentation
      doc: Improve GrlMedia documentation
      doc: Add GrlLog documentation
      doc: Fix GrlMediaVideo typo
      doc: Skip GrlDataSync
      doc: Added GrlError documentation
      doc: Fix GrlUtil documentation
      build: Define major, minor and micro version vars
      build: Fixed gir build rules
      vala: Improved metadata
      core: Moved tests infrastructure to /tests
      core: Updated infrastructure to support python tests
      tests: Command line arguments can be passed to testrunner
      tests: Tested the PluginRegistry class using GI
      tests: Tested the PluginMedia class using GI
      tests: Tested the MetadataSource class using GI
      tests: Removed setUp/tearDown code in python tests
      annotations: removed transfer mode for 'in' args
      annotations: Fixed transfer modes in grl-metadata-source
      core: reworked GrlMetadataSource's filter methods
      tests: updated GrlMetadataSource filter tests
      annotations: fixed GrlConfig annotations
      python: added plugin configuration to grilo-test-ui
\end{verbatim}

There were also minor contributions to the Grilo plugins

\begin{verbatim}
Simon Pena (4):
      vimeo: Updated calls to grl-media-video
      flickr: Removed const qualifier from grl-config
      bookmarks: Updated call to grl-media
      filesystem: Updated call to grl-media
\end{verbatim}

and to GTK-Doc (\href{http://git.gnome.org/browse/gtk-doc/commit/?id=50636742ee71fabd6fc306c4da53d2c431497f44}{Fix 'scope notified' annotation}), GObject Introspection
(\href{http://git.gnome.org/browse/gobject-introspection/commit/?id=ac29006d8481b8c8900672cfae834c12712ff2e6}{gio-2.0.c: Add missing annotations}) and PyGObject
(\href{http://git.gnome.org/browse/pygobject/commit/?id=28ed01c34c503cfb4f14fe7af7912060ca70aba6}{Don't check the inner type when comparing gpointers}).
During this work, several bugs were filed, all of them being \href{https://bugzilla.gnome.org/buglist.cgi?bug_status=RESOLVED,VERIFIED,CLOSED&emailreporter1=1&emailtype1=exact&email1=spenap%40gmail.com}{fixed} save
for the \href{https://bugzilla.gnome.org/show_bug.cgi?id=626047}{ParamSpec one} in Gjs.

\section{Conclusions and future work}

Automatic bindings using GObject Introspection are a really promising project. The
number of possibilities they can offer is just huge --just imagine developing a
Django Web Application which could use, natively, all the plugins and
functionality that Grilo offers!

However, while GObject Introspection is really interesting, it still feels young. The
number of languages offering good support for it is still small and there are still
important changes in its behaviour: recently, the default {\it transfer mode} stopped
being {\it full}, requiring library developers to explicit state it. As a result,
many projects stopped offering automatic bindings. While the decision is easy to
understand (requiring library developers to state which {\it transfer mode} they need
guarantees that there will not be leaks nor wrong frees), it serves as an example of
how unstable the project is at the moment.

With regards to collaborating on a Free Software project, contributing to Grilo has
been a great experience. I became familiar with the tools like gdb, to be able to debug
both C + GObject code or to locate a estrange crash in Python invoking Grilo methods;
valgrind, to check if I was having a leak due to incorrectly freeing memory; JHBuild, to
build a GNOME environment in a sandbox, using upstream versions; git, to create branches,
apply patches, check the history, locate an error thanks to {\it git bisect}\dots

I also felt comfortable with languages I had not used too much, such as JavaScript or Python,
learning about unit testing with the latter, and continued learning about C + GObject.

Improving Grilo's support of GObject Introspection also gave me the chance to keep
an eye on other interesting projects such as PyGObject, Gjs or GTK-Doc. I had the opportunity
to find and file bugs on them, sometimes even providing patches.

As for the future work, while it should be important to keep the GObject Introspection support
updated (checking changes related to default parameters or new features), it would also be
interesting to provide new plugins. Improving the existing Python unit tests would be useful
and creating Gjs' ones would allow a better evaluation of that language's support. Seed's
matureness, compared to Gjs, should be checked, and a better and larger test in Vala should
definitely be added, too.
