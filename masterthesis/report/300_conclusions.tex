This report has covered all the work done during the third edition of the
Master on Free Software.

First, the background and motivations leading to the enrolment were stated,
and a set of objectives was introduced to allow a proper self-evaluation once
that the master was over. The evaluation is really satisfactory, but a more
detailed one will be done in the last section of this chapter.

Then, the subjects covered during the master were reviewed and the work done
on each area, both theoretical and practical, was presented. While the fields
of knowledge covered were really broad, all the subjects were really enjoyable.
The next section will present some specific conclusions about them.

After that, the master's {\it practicum} work done in the Grilo project at Igalia
was explained, as well as other contributions originated from that work. More complete
conclusions will be presented in the second section of this chapter.

\section{Subjects}

The subjects taken during the master covered a really wide range of topics.
While my main interests were on the development modules, I was happily surprised
with a subject I had not think of: Software Engineering techniques applied
to {\it Libre Software} projects. 

There are really large {\it Libre Software} projects (such as the Linux kernel,
GNOME, KDE or many others) with plenty of open information available (software code repository,
bugs databases, mailing lists archives\dots), being possible to
analyze all these data sources, getting to know the health of the project,
who the main developers are, how bugs are distributed between releases, how
active the community around the project is\dots But not only can
{\it Libre Software} projects benefit from Software Engineering techniques,
but also those techniques can be improved. Accessing to proprietary software
code to perform certain analysis was expensive and required researchers to
sign NDAs, and that would cause those analysis to be impossible to replicate,
banning peer review. As a consequence, I felt like doing Software Engineering analysis
on {\it Libre Software} projects was not only an innovative approach but also
a logical one.

\subsection{FOSDEM}

As a complementary activity which could be considered part of the {\it Introduction
to Libre Software} and {\it Detailed Studies of Libre Software Communities}, the
students were sponsored to go to the FOSDEM in Brussels: The Free and Open source
Software Developers' European Meeting, one of the most important
{\it Libre Software} conferences. During a weekend in February we had the opportunity
to attend to several events: from a Beer event, where developers would meet and
have a beer, to lots of different keynotes, passing through a key-signing party.

Meeting the key developers of many important projects and communities, listening to
them talking about their work or just enjoying the ambient: it was an unforgettable
experience.

\subsection{GUADEC-ES}

After the work done on the Desktop \& Mobile development project, and taking
advantage from the GUADEC-ES being in A Coru\~na, we were suggested to submit
a talk proposal.

The GUADEC-ES is the main GNOME conference in Spanish language. In its 7th edition,
it was based on A Coru\~na, which made easy for us attending to it. I was responsible
of elaborating the talk proposal, explaining the development done in the Desktop \&
Mobile application.

As the proposal got approved, I went to write the presentation slides and give the
talk. It was a really satisfactory experience, even if I felt that we still had
a lot of work to do to get close to other GNOME {\it hackers} --and become {\it hackers}
ourselves. Nevertheless, the conference was very interesting and people there was
helpful and willing to share what they knew, so it was a perfect complement to the
development courses.

\section{Practicum}
During the {\it practicum}, I had the opportunity to put in use all the knowledge acquired
during the master's subjects. I employed technologies such as git, for source code
versioning and generating patches for peer reviews, and did extensive use of the
mailing list and IRC to communicate with other Grilo developers (setting up
a \href{http://simonpena.homedns.org/grilo-logs/}{server} for making Grilo IRC logs available). 

As a consequence of using so {\it cutting-edge} technologies, it was common to find
issues either in GObject Introspection itself or in PyGObject, Gjs or even GTK-Doc
(as it needs to recognize the annotations used in GObject Introspection). That led me to
identify, file and sometimes fix bugs present in these projects, while communicating
with other upstream {\it hackers}. All this experience was really rewarding, and
as a result of it, I also got committer permissions to the GNOME project.

\section{Summary}

As a final summary of this report, a review of each individual is provided:

\begin{description}
\item {Formalizing Knowledge:} While initially I only focused on formalizing
my systems administration knowledge, now that the master is over, I feel like
I have consolidated all fields studied. From said systems administration to programming,
passing trough licensing or Software Engineering, I now have a really broad vision
of these areas. I have discovered new sources for further learning, good channels
to ask to experts and a number of contacts in the field who will surely prove
helpful. A good example for that already happened during the GUADEC-ES, when
well-known {\it hackers} Javier Jard\'on and Nacho (Ignacio Casal Quinteiro) helped
me solve some JHBuild issues.

\item {Mobile and Desktop Development}: As I have hoped, mobile development was
focused in a way which allowed reusing of existing components and technologies,
also present in desktop developments. We enjoyed Maemo and Android sessions, and
were also told about MyMobileWeb, a project which targets mobile devices from
the web development point of view. Accomplishing this objective was really
rewarding: besides the practical development being done in the master scope,
I could give a great push to a personal project of mine, Butaca, which follows
the same architectonic design principles. Having the opportunity
to attend to FOSDEM and GUADEC-ES and giving a presentation in the later was
simply great.

\item {Desktop Development}: The existing collaboration between KDE and GNOME
communities is really a relief, and should work fine against the typical
said that they are fighting each other and wasting resources. The freedesktop
project, being a strong {\it foundation}\footnote{As in buildings} on which both
projects can rely is a great example on how collaboration and competition are
the engine which moves {\it Libre Software}

\item {Licensing:} As I previously said with regards to {\it formalizing knowledge},
I know feel like I have consolidated my knowledge of the licensing field. I can now
tell from different licenses, classify them between {\it permissive} and {\it copyleft},
know what are these kind of licenses better for, and how they interact with
each other. With this information, I can also help deciding which license fits better
a new project in an area where it is being innovative, or which one is more
suitable when there are already other options.

\item {Role Played:} My first contribution to a {\it Libre Software}
project got accepted quite early during the master, being a patch to mlstats which
improved the FROM field extraction when parsing mailing lists. After that,
I contributed with many improvements to Grilo in the {\it practicum} scope,
from enhancements in the automatic binding generation to code optimizations or other
fixes. While doing the contributing to Grilo, I also had the chance to file bugs
and provide patches to different issues in other projects, such as GTK-Doc, PyGObject,
or GObject-Introspection.
\end{description}
