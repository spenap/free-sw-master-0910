\documentclass{beamer}

\usepackage[english]{babel}
\usepackage[latin1]{inputenc}

%\usepackage{pgf}
\usepackage{color}
\usepackage{multimedia}
\usepackage{eurosym}

% Estos son los temas b·sicos (hay variantes 
% en cada uno de ellos). Descomenta uno y compila.
%\usetheme{Goettingen}
%\usetheme{Warsaw}
\usetheme{Copenhagen}
%\usetheme{Dresden}
%\usetheme{Singapore}
%\usetheme{Szeged}
%\usetheme{Pittsburgh}
%\usetheme{Malmoe}
%\usetheme{Montpellier}

% Algunos temas de colores est·ndar (son feos!)
%\usecolortheme{albatross}
%\usecolortheme{beetle}
%\usecolortheme{crane}

% Este comando hace los puntos redonditos
\beamertemplateballitem

% Esta opciÛn sirve para que, cuando pasamos la
% presentaciÛn con pausas, lo que estÈ oculto 
% se vea de color muy clarito
\setbeamercovered{transparent}
\setbeamertemplate{navigation symbols}{}

% DeclaraciÛn de im·genes a usar 
\pgfdeclareimage[width=50px]{cc-license}{images/cc-by-sa.png}


% TÌtulo, subtÌtulo, autor, afiliaciÛn y fecha
\title{Open Source Software \& Software Devices}
\subtitle{Introduction to Libre Software}
\author{Sim\'on Pena Placer}
\institute{M\'aster Software Libre, 2009-2010. A Coru\~na Edition}
\date{\today}


\begin{document}


% Portada
\begin{frame}
	\titlepage
\end{frame}

\begin{frame}
	\frametitle{License}
	\begin{center}
	\pgfuseimage{cc-license}
	\end{center}
This work is licensed under the Creative Commons Attribution-Share Alike 3.0 Spain License. To view a copy of this license, visit \url{http://creativecommons.org/licenses/by-sa/3.0/es/} or send a letter to Creative Commons, 171 Second Street, Suite 300, San Francisco, California, 94105, USA.
\end{frame}

% Õndice
\begin{frame}
	% Con este comando se pone el tÌtulo de la diapositiva
	\frametitle{Content}
	\tableofcontents
\end{frame}

% Con esto hacemos que al entrar en cada secciÛn aparezca el Ìndice
% con la secciÛn actual remarcada
% TambiÈn es posible la opciÛn \AtBeginSubSection
\AtBeginSubsection[]
{
  \begin{frame}<beamer>
    \frametitle{Outline}
    \tableofcontents[current,currentsubsection]
  \end{frame}
}

\section{Introduction}

\begin{frame}{Introduction}
\begin{itemize}
\item This work is a small review of the current state of the art regarding Open Source-related Operating Systems for mobile devices
\item Operating Systems will be chosen both for their market share or their relevance
\end{itemize}
\end{frame}

\section{Mobile Operating Systems}

\begin{frame}{Symbian}
\begin{itemize}
\item Symbian is still the most used mobile operating system
\item Since the iPhone launch, Symbian is losing track
\item It was the OS of choice for Nokia, Motorola, Sony Ericsson, Siemens\dots
\begin{itemize}
\item Core development was centralized
\item User Interfaces were adapted and customized for each manufacturer.
\end{itemize}
\item It is currently being opened under the Symbian Fundation
\begin{itemize}
\item Code will be available during the process royalty free for members
\item It will be free for all once the process is over
\item Their selected license will be the Eclipse Public License
\end{itemize}
\end{itemize}
\end{frame}

\begin{frame}{Android}
\begin{itemize}
\item Software stack for mobile devices including operating system, middleware and key applications
\item Initially developed by Google, later by the Open Handset Alliance
\item Licensed under the Apache Software License
\item Some criticism exists due to the possibility for manufacturers to hold their changes and modifications due to the license chosen
\end{itemize}
\end{frame}

\begin{frame}{Maemo}
\begin{itemize}
\item Software platform mostly based on open source code
\item Developed in collaboration with projects such as the LInux kernel, Debian, Gnome and many others
\item Mostly a GNU / Linux distribution with closed proprietary software from Nokia and a phone stack
\item While not being used in the current version, plans are that Nokia and Intel will start using oFono, and open source telephony stack
\item Derivative project Mer aims to provide a fully open source system
\end{itemize}
\end{frame}

\begin{frame}{Moblin}
\begin{itemize}
\item Targets mobile devices such as netbooks and MIDs 
\item Built around the Intel Atom Processor
\item Designed to minimize boot times and power consumption.
\end{itemize}
\end{frame}


\begin{frame}{WebOS}
\begin{itemize}
\item Based on a Linux 2.6 kernel
\item Runs several open source components
\item Applies a proprietary layer over them, referred to as the Core OS
\item Based on WebKit
\item Development done with HTML, CSS and JavaScript
\end{itemize}
\end{frame}

\begin{frame}{Openmoko}
\begin{itemize}
\item Designed to provide end users with the ability to modify the operating system and software stack
\item Originally focused at providing both hardware and software
\item Focused on software nowadays
\end{itemize}
\end{frame}

\begin{frame}{Bada}
\begin{itemize}
\item New smartphone operating system from Samsung
\item Targeted at low-cost phones
\item Uses terms like ``open'' and ``free'' but no info has been shown yet
\item It is said to be able to run on Linux kernels
\item Samsung is still present in the Symbian Foundation, LiMo Foundation and the Open Handset Alliance
\end{itemize}
\end{frame}

\begin{frame}{iPhone OS}
\begin{itemize}
\item Built on top of the Darwing Core (Open Source BSD kernel), unavailable for other platforms than x86
\item Interesting relationship with Open Source third party applications
\begin{itemize}
\item Apple SDK's terms of use put a NDA on the APIs, limiting developers ability to release code
\item NDA was later removed
\item Apple's private keys for signing still put problems to licenses like GPLv3
\end{itemize}
\end{itemize}
\end{frame}

\begin{frame}{LiMo}
\begin{itemize}
\item LiMo Foundation is an industry consortium using a Linux-based operating system.
\item Platform consisting of Member contributions and Open Source technologies
\item Open development, but conditioned to the OEM decisions
\end{itemize}
\end{frame}

\begin{frame}{Mobilinux}
\begin{itemize}
\item Operating system and development platform.
\item Seems a low level layer to build under LiMo, Android, other systems.
\end{itemize}
\end{frame}

\section{Conclusions}

\begin{frame}{Conclusions}

\begin{itemize}
\item Open Source Software is getting a great push in the Mobile Devices Market
\item Linux Kernel is one of the most used solutions, used in WebOS, Android, Maemo, OpenMoko, LiMo, Mobilinux
\item Shared development costs is one of the main reasons to use Open Source solutions
\item Proprietary solutions are still used for specific components.
\item Openness during development is not likely to affect and benefit end-users
\end{itemize}

\end{frame}

% Portada
\begin{frame}
	\titlepage
\end{frame}

\end{document}

