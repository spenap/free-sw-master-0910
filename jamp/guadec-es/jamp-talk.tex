\documentclass{beamer}

\usepackage[spanish]{babel}
\usepackage[utf8x]{inputenc}

\usepackage{color}
\usepackage{multimedia}
\usepackage{eurosym}

\usetheme{Dresden}

\beamertemplateballitem

\setbeamercovered{transparent}
\setbeamertemplate{navigation symbols}{}

\pgfdeclareimage[width=35px]{cc-license}{images/cc-by-sa.png}
\pgfdeclareimage[width=35px]{igalia-logo}{images/igalia-logo.pdf}
\pgfdeclareimage[width=35px]{libresoft-logo}{images/libresoft-logo.jpg}
\pgfdeclareimage[width=250px]{jamp-architecture}{images/jamp-architecture.pdf}
\pgfdeclareimage[width=150px]{jamendo-logo}{images/jamendo-logo.pdf}
\pgfdeclareimage[width=90px]{jamp-screenshot}{images/jmp-gnome-search.png}
\pgfdeclareimage[width=90px]{jamp-maemo-screenshot}{images/jmp-maemo-search-results.png}
\pgfdeclareimage[width=150px]{jamp-maemo-home}{images/jmp-maemo-home-screen.png}
\pgfdeclareimage[width=150px]{jamp-maemo-search}{images/jmp-maemo-search-screen.png}
\pgfdeclareimage[width=125px]{maemo-logo}{images/maemo_org_logo.png}
\pgfdeclareimage[width=125px]{nokia-n900}{images/N900.png}
\pgfdeclareimage[width=220px]{people}{images/mswl-people.jpg}

\setbeamertemplate{sidebar left}{%
\vfill%
\rlap{\hskip0.1cm%
{\pgfuseimage{cc-license}}
\hskip0.1cm
{\pgfuseimage{libresoft-logo}}
\hskip0.1cm
{\pgfuseimage{igalia-logo}}
}
\vskip2pt%
\llap{\usebeamertemplate***{navigation symbols}\hskip0.1cm}%
\vskip2pt%
}

\title{JaMp - Un cliente para Jamendo}

\author[Simón Pena]{Simón Pena y Alumnos del Máster en Software Libre}
\institute[Máster en Software Libre.
Edición de A Coruña. Igalia -- URJC]
{Máster en Software Libre, 2009-2010. Edición de A Coruña\\
Igalia - Universidad Rey Juan Carlos}
\date{GUADEC-ES -- 23 de julio de 2010}

\begin{document}

% Portada
\begin{frame}
\titlepage
\end{frame}

% Restablecemos la plantilla para eliminar los logos de la portada en adelante
\setbeamertemplate{sidebar left}{}

\begin{frame}{Licencia}
Esta obra está bajo una licencia Attribution-ShareAlike 3.0 Spain de Creative
Commons. Para ver una copia de esta licencia, visite
\url{http://creativecommons.org/licenses/by-sa/3.0/es/}
o envie una carta a Creative Commons, 171 Second Street, Suite 300, San Francisco, California 94105, USA.
\end{frame}

% Índice
\begin{frame}{Contenido}
\tableofcontents
\end{frame}

\AtBeginSubsection[]
{
\begin{frame}<beamer>{Contenido}
\tableofcontents[current,currentsubsection]
\end{frame}
}

\section{Introducción}

\subsection{¿Qué es?}

\begin{frame}{¿Qué es JaMp?}
\begin{block}{JaMp}
\begin{itemize}
  \item Ejercicio práctico en el módulo de desarrollo del Máster Software Libre
  \item Cliente para Jamendo para la plataforma GNOME
  \item Git como sistema de control de versiones
  \item Envío e integración de parches
  % listas de correo para la comunicación
  % interna, así como para compartir e integrar parches, e IRC para la
  % comunicación más instantánea.
\end{itemize}
\end{block}
\end{frame}

\begin{frame}{¿Qué es Jamendo?}
\begin{center}
\pgfuseimage{jamendo-logo}
\end{center}
\begin{itemize}
  \item Comunidad de música libre, legal e ilimitada, publicada bajo licencias
  Creative Commons
  \item 36299 álbumes publicados
  \item 226880 Reseñas de álbumes
  \item 809051 miembros activos
\end{itemize}
\end{frame}

\begin{frame}{¿Por qué JaMp?}
\begin{block}{Intereses didácticos}
\begin{itemize}
  \item Punto de entrada a la plataforma
  \item Familiarizarse con tecnologías habituales
  \begin{itemize}
    \item Acceso y manipulación de contenidos web
    \item Reproducción de los contenidos multimedia
  \end{itemize}
\end{itemize}
\end{block}
\begin{block}{Intereses prácticos}
\begin{itemize}
  \item Cliente dedicado (frente a usar plugins)
  \item Arquitectura solida y modular
\end{itemize}
\end{block}
\end{frame}

\subsection{Arquitectura}

\begin{frame}{Arquitectura utilizada}
\begin{center}
\pgfuseimage{jamp-architecture}
\end{center}

% Go to: 
% http://yuml.me/diagram/scruffy/class/edit/
%
% and paste following code uncommented: 
%
%[JaMp Frontend (Python + GTK)]-.->[Jamp Controller (D-Bus)],[JaMp Frontend
%(Python + GTK)]->[Jamp Backend (C + GObject)],[JaMp Controller (D-Bus)]->[JaMp
%Frontend (Python + GTK)],[JaMp Controller (D-Bus)]->[JaMp Backend (C +
%GObject)],[JaMp Backend (C + GObject)]-.->[JaMp Frontend (Python + GTK)]

\begin{block}{Patrón arquitectónico Model-View-Controller}
\begin{itemize}
  \item Un {\it backend} para la recuperación de información de Jamendo y la
  reproducción multimedia
  \item Un {\it frontend} para la presentación de la información obtenida y la
  gestión del la reproducción
  \item Interconexión mediante D-Bus
\end{itemize}
\end{block}
\end{frame}

\begin{frame}{Backend}
\begin{block}{Desarrollado en C con GObject}
\begin{itemize}
  \item Un proveedor de datos
  \begin{itemize}
    \item Accede a Jamendo usando libsoup
    \item Manipula los datos recibidos usando libxml
    \item Expone un método {\it Query}
  \end{itemize}
  \item Un gestor multimedia
  \begin{itemize}
    \item Permite la reproducción de los contenidos usando gstreamer
    \item Expone métodos {\it Play}, {\it Pause}, notificación de progreso\ldots
  \end{itemize}
\end{itemize}
\end{block}
\end{frame}

\begin{frame}{Frontend}
\begin{center}
\begin{figure}
\pgfuseimage{jamp-screenshot}
\hskip0.1cm
\pgfuseimage{jamp-maemo-screenshot}
\caption{Clientes buscando ``rock''}
\end{figure}
\end{center}
\begin{block}{Desarrollado en Python}
\begin{itemize}
  \item Un cliente para GNOME, desarrollado con Glade
  \item Un cliente para Maemo, usando Hildon
\end{itemize}
\end{block}
\end{frame}

\begin{frame}{Conexión mediante D-Bus}
\begin{itemize}
  \item Se usa D-Bus GLib en el lado del backend
  \item Todas las llamadas son asíncronas
  \item El modelo está totalmente desacoplado de la vista
  \item El controlador se puede reaprovechar entre distintas vistas
\end{itemize}
\end{frame}

\section{Desarrollando JaMp}
% * El port a Fremantle (Maemo 5)
%
% Gracias a la arquitectura de la aplicación, para adaptar la misma a
% Fremantle sólo es necesario crear una nueva interface de usuario usando
% Hildon, pudiéndose reutilizar sin cambios tanto los backends como el
% servicio de D-Bus -aunque una posible mejora pasa por sustituir el módulo
% de gestión multimedia por el propio de Maemo 5, MAFW (Multimedia
% Application Framework). El uso de Hildon permite emplear widgets que
% optimizan el tamaño de pantalla y permiten una interacción con el
% usuario de un modo "finger-friendly".
%
% La comunicación con D-Bus también se encuentra desacoplada de la vista,
% lo que facilitará la creación de nuevas interfaces de usuario.
%
% * Otras cuestiones de interés
%
% Durante el desarrollo, también se han tratado aspectos como la
% accesibilidad, la internacionalización y la localización, la
% documentación -tanto a nivel de usuario como de desarrolladores-, las
% pruebas de unidad u otras cuestiones como la compilación condicional
% (por ejemplo, para facilitar ports a otras plataformas)


\subsection{Manejando señales}

\begin{frame}[fragile]{Manejando señales (I)}
\begin{block}{Declarando la señal}

\small{Se instala en la inicialización de la clase ({\it
pref\_foo\_class\_init})}

\scriptsize{
\begin{verbatim}
jmp_mplayer_signals[ERROR] = g_signal_new ("error",
                                G_TYPE_FROM_CLASS (klass),
                                G_SIGNAL_RUN_LAST,
                                0,
                                NULL,
                                NULL,
                                g_cclosure_marshal_VOID__STRING,
                                G_TYPE_NONE,
                                1,
                                G_TYPE_STRING);
\end{verbatim}
}
\end{block}
\end{frame}

\begin{frame}[fragile]{Manejando señales (y II)}
\begin{block}{Emisión}

\small{Se envía la señal de acuerdo a la lógica de la aplicación (aquí, al
recibir un error en {\it gstreamer})}

\scriptsize{
\begin{verbatim}
g_signal_emit (self, jmp_mplayer_signals[ERROR], 0, error_message);
\end{verbatim}
}
\end{block}
\begin{block}{Conexión}

\small{Nos conectamos a la señal en el cliente que use el componente (que podrá
decidir qué comportamiento aplicar en función de los valores recibidos)}

\scriptsize{
\begin{verbatim}
g_signal_connect (jmplayer, "error", G_CALLBACK (error_callback), NULL);
\end{verbatim}
}
\end{block}
\end{frame}

\begin{frame}{Marshallers y GClosures (I)}

\begin{itemize}
  \item Los {\it GClosure}s permiten representar funciones de callback.
  \item Existe un número de {\it GClosure}s predefinidos
  \begin{itemize}
    \item g\_cclosure\_marshal\_VOID\_\_VOID 
    \item g\_cclosure\_marshal\_VOID\_\_BOOLEAN
    \item \ldots
  \end{itemize}
  \item Es habitual querer callbacks no predefinidos
  \uncover<2->\textbf{¿Cómo?}
  \item<3-> Usando {\it glib-genmarshal}
\end{itemize}

\end{frame}

\begin{frame}[fragile]{Marshallers y GClosures (II)}

\textbf{glib-genmarshal}

\begin{itemize}
  \item Permite crear GClosures personalizados
  \item Recibe como entrada una lista $RETURN\_VALUE:ARG_{1},ARG_{i},ARG_{N}$
\end{itemize}
\begin{block}{Fichero marshal.list}
\begin{verbatim}
VOID:INT64,INT64
VOID:STRING,POINTER
VOID:STRING,UINT,POINTER
\end{verbatim}
\end{block}
\end{frame}

\begin{frame}[fragile]{Marshallers y GClosures (y III)}

\begin{block}{Soporte en autotools}
\scriptsize{
\begin{verbatim}
jmp-marshal.h: marshal.list
	glib-genmarshal --header --prefix=jmp_marshal marshal.list > $@

jmp-marshal.c: marshal.list
	glib-genmarshal --body --prefix=jmp_marshal marshal.list > $@
\end{verbatim}
}
\end{block}

\begin{block}{Usándolo en una señal}
\scriptsize{
\begin{verbatim}
..., jmp_marshal_VOID__INT64_INT64, G_TYPE_NONE, 
     2, G_TYPE_INT64, G_TYPE_INT64, NULL);
     
static void
tick_cb (JmpMplayer *jmplayer, gint64 position, gint64 duration,
               gpointer user_data)
\end{verbatim}
}
\end{block}
\end{frame}

\subsection{Accediendo a servicios web}

\begin{frame}[fragile]{Accediendo a Jamendo (I)}{Usando libsoup}
\begin{block}{Encolando un mensaje}

\small{Encolamos una petición para comunicarnos de forma asíncrona}
\scriptsize{
\begin{verbatim}
...
soup_session_queue_message (session, msg, process_response, cbdata);
...
\end{verbatim}
}
\end{block}

\begin{block}{Recibiendo la respuesta}
\small{Una vez recibida la respuesta, se gestionará aquí}
\scriptsize{
\begin{verbatim}
static void
process_response (SoupSession *session, SoupMessage *msg, 
    gpointer user_data);
\end{verbatim}
}
\end{block}
\end{frame}
% 
\begin{frame}[fragile]{Accediendo a Jamendo (II)}{Analizando la respuesta:
usando libxml2}
\scriptsize{
\begin{verbatim}
static GList*
parse_xml (const char *buffer, int length, JmpRelation relation)
{
        xmlDocPtr doc = xmlReadMemory (buffer, length, NULL, NULL,
                                       XML_PARSE_NOBLANKS | XML_PARSE_RECOVER);
        if (!doc) return NULL;

        GList *result = NULL;

        xmlXPathContextPtr context = xmlXPathNewContext (doc);

        /* xpath is of the form "/data/album" */
        xmlXPathObjectPtr xpath_obj =
                xmlXPathEvalExpression (vos[relation].xpath, context);
        if (!xpath_obj) goto bail;
        ...
\end{verbatim}
}
\end{frame}

\begin{frame}[fragile]{Accediendo a Jamendo (y III)}{Analizando la respuesta:
usando libxml2}
\tiny{
\begin{verbatim}
    xmlNodeSetPtr nodeset = xpath_obj->nodesetval;
    if (nodeset->nodeNr > 0)
            result = generate_list (nodeset, relation);
    
    /* generate_list */
    for (i = 0; i < nodes->nodeNr; i++) {
            xmlNodePtr node = nodes->nodeTab[i];
            if (node->type == XML_ELEMENT_NODE) {
                    GObject *item = generate_vo (node->children, relation);
                    if (item)
                            list = g_list_prepend (list, item);
            }
    }
    
    /* generate_vo */
    for (cur_node = a_node; cur_node; cur_node = cur_node->next) {
             if (cur_node->type == XML_ELEMENT_NODE) {
                     char *value = xmlNodeGetContent(cur_node);
                     item_set (item, cur_node->name, value, relation);
\end{verbatim}
}
\end{frame}

\subsection{Utilizando D-Bus}

\begin{frame}{Utilizando D-Bus}
\begin{block}{Pasos}
\begin{itemize}
  \item Crear una interfaz XML exponiendo los métodos
  \item Crear un servicio que implemente la interfaz, empleando D-Bus GLib
  \item Integrar en autotools
  \item Acceder desde Python
\end{itemize}
\end{block}
\end{frame}

\begin{frame}[fragile]{Utilizando D-Bus (II)}{Definiendo una interfaz}
\tiny{
\begin{verbatim}
<?xml version="1.0" encoding="UTF-8" ?>
<!DOCTYPE node PUBLIC
	  "-//freedesktop//DTD D-Bus Object Introspection 1.0//EN"
	  "http://standards.freedesktop.org/dbus/1.0/introspect.dtd">
<node name="/">
  <interface name="org.mswl.jamp">
    <method name="Play">
    </method>
    <method name="GetVolume">
      <arg type="d" name="volume" direction="out" />
    </method>
    <method name="Seek">
      <arg type="x" name="seek_position" direction="in" />
    </method>
    <signal name="tick">
      <arg type="x" name="position" />
      <arg type="x" name="duration" />
    </signal>
  </interface>
</node>
\end{verbatim}
}
\end{frame}

\begin{frame}[fragile]{Utilizando D-Bus (III)}{Implementando la interfaz}
\tiny{
\begin{verbatim}
#include <dbus/dbus-glib.h>
...
connection = dbus_g_bus_get (DBUS_BUS_SESSION, &error);
...
#include <dbus/dbus-glib-bindings.h>
...
gboolean
jmp_mplayer_service_seek (JmpMplayerService *self, gint64 seek_position,
                          GError **error)
{
        return jmp_mplayer_seek (self->priv->player, seek_position);
}
...
gboolean
jmp_mplayer_service_get_volume (JmpMplayerService *self, double *volume,
                                GError **error)
{
        *volume = jmp_mplayer_get_volume (self->priv->player);
        return TRUE;
}
\end{verbatim}
}
\end{frame}

\begin{frame}[fragile]{Utilizando D-Bus (III)}{Configurando D-Bus}
\scriptsize{
\begin{verbatim}
...
DBusGProxy *proxy = dbus_g_proxy_new_for_name (self->priv->connection,
                                   DBUS_SERVICE_DBUS,
                                   DBUS_PATH_DBUS,
                                   DBUS_INTERFACE_DBUS);
...
org_freedesktop_DBus_request_name (proxy,
                                MPLAYER_SERVICE_NAME,
                                0, &request_name_result,
                                &error));
...
dbus_g_connection_register_g_object (self->priv->connection,
                                     MPLAYER_SERVICE_OBJECT_PATH,
                                     G_OBJECT (self));
\end{verbatim}}
\end{frame}

\begin{frame}[fragile]{Utilizando D-Bus (y IV)}{Conectando desde Python}
\tiny{
\begin{verbatim}
bus = dbus.SessionBus()

class PlaybackManager:
    def __init__(self, tick_cb=None):
        self._bus = bus
        self._tick_cb = tick_cb
        player_object = self._bus.get_object("org.mswl.jamp",
                                             "/Player")
        self._player_interface = dbus.Interface(player_object,
                                          dbus_interface="org.mswl.jamp")
        if self._tick_cb:
            self._player_interface.connect_to_signal("tick",
                                                     self._tick_cb)
    def seek(self, value):
        self._player_interface.Seek(value)
...
def tick_signal_handler(self, position, duration):
    ...
\end{verbatim}
}
\end{frame}

\subsection{Port para Maemo}

\begin{frame}[fragile]{Port para Maemo}{¿Qué es Maemo?}

\begin{center}
\pgfuseimage{maemo-logo}
\hskip0.4cm
\pgfuseimage{nokia-n900}
\end{center}

\begin{itemize}
  \item Plataforma que corre en dispositivos móviles de Nokia
  \begin{itemize}
    \item Maemo 3: Internet Tablet 770
    \item Maemo 4: Internet Tablets N800 y N810
    \item Maemo 5: N900
  \end{itemize}
  \item Alrededor de los dispositivos se forma la comunidad Maemo
\end{itemize}
\end{frame}

\begin{frame}[fragile]{Port para Maemo}{¿En qué fijarse?}
\begin{block}{Se utiliza Hildon}
\scriptsize{
\begin{verbatim}
hildon.StackableWindow
hildon.GtkButton(gtk.HILDON_SIZE_AUTO)
hildon.PickerButton(gtk.HILDON_SIZE_FINGER_HEIGHT,
                    hildon.BUTTON_ARRANGEMENT_HORIZONTAL)
\end{verbatim}
}
\end{block}
\pause
\begin{block}{La aplicación no debe parecer bloqueada}
\scriptsize{
\begin{verbatim}
hildon.hildon_gtk_window_set_progress_indicator(self, True)
banner = hildon.hildon_banner_show_information(self, "", message)
banner.set_timeout(milliseconds)
\end{verbatim}}
\end{block}
\end{frame}

\begin{frame}{Port para Maemo}{Claridad y sencillez, ``finger friendly''}
\begin{center}
\begin{figure}
\pgfuseimage{jamp-maemo-home}
\hskip0.1cm
\pgfuseimage{jamp-maemo-search}
\caption{Pantalla de bienvenida y búsqueda}
\end{figure}
\end{center}
\end{frame}

\section{Futuro}

\subsection{Objetivos}

\begin{frame}{Objetivos futuros (I)}
\begin{block}{Distribución}
\begin{itemize}
  \item Empaquetar JaMp
  \item Crear un repositorio {\it PPA} para Ubuntu
\end{itemize}
\end{block}
\begin{block}{Interacción con Jamendo}
\begin{itemize}
  \item Aumentar las APIs implementadas
  \item Emplear Grilo para el acceso a los datos
\end{itemize}
\end{block}
\end{frame}

\begin{frame}{Objetivos futuros (y II)}
\begin{block}{Ports}
\begin{itemize}
  \item Completar port para Maemo
  \item Emplear MAFW en Maemo
  \item Llevar a otras plataformas
\end{itemize}
\end{block}
\begin{block}{Otras posibilidades}
\begin{itemize}
  \item Emplear otros toolkits gráficos
\end{itemize}
\end{block}
\end{frame}

\subsection{¡Únete!}

\begin{frame}{¡Únete a nosotros!}
\begin{block}{¿Por qué?}
\begin{itemize}
  \item Te interesa Jamendo
  \item Te interesa conocer las tecnologías que usamos
  \item Te interesa llevar JaMp a una nueva plataforma
  \item Quieres corregir algún error que hayas visto hoy ;)
\end{itemize}
\end{block}
\begin{description}
\item[Descarga el código] \url{http://gitorious.org/mswl2010/jamp}
\item[Ponte en contacto con nosotros] \url{jamp-devel@googlegroups.com}
\end{description}
\end{frame}

\begin{frame}{¡¡Gracias!!}
\begin{center}
\pgfuseimage{people}
\end{center}
\end{frame}

\begin{frame}{Referencias}
\scriptsize{
\begin{description}
\item[Señales]
\url{http://library.gnome.org/devel/gobject/stable/signal.html}
\item[D-Bus] \url{http://www.freedesktop.org/wiki/Software/dbus}
\item[Hildon] \url{http://wiki.maemo.org/Hildon}
\item[PyMaemo] \url{http://pymaemo.garage.maemo.org/}, \url{http://wiki.maemo.org/PyMaemo/UI_tutorial}
\end{description}}
\end{frame}

\end{document}
