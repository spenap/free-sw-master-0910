% Definimos el estilo del documento
\documentclass[12pt]{article}
\usepackage[spanish]{babel}

% Acentos y e�es
\usepackage[latin1]{inputenc}
\usepackage{url}
\usepackage{hyperref}

% Definimos titulo, autor, fecha.
\title{Laboratorio III}
\author{Sim\'on Pena Placer}
\date{\today}

%Empieza el documento
\begin{document}

% Generamos titulo e indice de contenidos
\maketitle
% \tableofcontents

% Empezamos capitulos
\section{Ejercicio 1}

\textbf{�Qu� comandos ser�an necesarios ejecutar para que un sistema Linux pudiese sustituir el
encaminador R2 mostrado en el diagrama? Asume todos aquellos datos que necesites para
realizar el ejercicio (nombre de interfaces, gateway, etc)}

Asumimos que el sistema Linux tiene dos interfaces: ``eth0'' y ``eth1'', conectadas, respectivamente, a la {Red 2} y a la {\it Red 3}. En su tabla de rutas deber�n aparecer al menos tres entradas: una para cada una de las interfaces, y una tercera para el otro router visible, al que se enviar� el tr�fico que no corresponda a ninguna de las redes.

{\footnotesize
\begin{verbatim}
R2:~#ifconfig eth0 10.10.0.2\
>broadcast 10.10.0.255 netmask 255.255.255.0
R2:~#ifconfig eth1 200.3.107.1\ 
>broadcast 200.3.107.255 netmask 255.255.255.0
\end{verbatim}
}

Al configurar las interfaces, la tabla de ruta se actualiza, autom�ticamente, indicando que estamos conectados a las subredes mediante esas interfaces. Se conseguir�a el mismo efecto mediante estos comandos:

{\footnotesize
\begin{verbatim}
R2:~#route add -net 10.10.0.0 netmask 255.255.255.0 dev eth0
R2:~#route add -net 200.3.107.0 netmask 255.255.255.0 dev eth1
\end{verbatim}
}

Finalmente, a�adimos la puerta de enlace predeterminada

{\footnotesize
\begin{verbatim}
R2:~#route add default gw 10.10.0.1
\end{verbatim}
}

Ser�a necesario, posteriormente, a�adir reglas de {\it iptables} y activar {\it NAT}.

\section{Ejercicio 2}

\textbf{Configura un alias de interface con una IP libre en el rango de red que est� siendo usado por
tu interface inal�mbrica y realiza las siguientes pruebas:
\begin{itemize}
    \item Verifica que se encuentra activa
    \item Env�a un ping a uno de tus compa�eros y anota el resultado
\end{itemize}
}

{\footnotesize
\begin{verbatim}
spenap@mannin:~$ sudo ifconfig eth2:1 192.168.1.112 \
>broadcast 192.168.1.255 netmask 255.255.255.0
[sudo] password for spenap: 
spenap@mannin:~$ ifconfig 
eth0      Link encap:Ethernet  direcci�nHW 00:22:41:20:6e:bd  
          ACTIVO DIFUSI�N MULTICAST  MTU:1500  M�trica:1
          Paquetes RX:0 errores:0 perdidos:0 overruns:0 frame:0
          Paquetes TX:0 errores:0 perdidos:0 overruns:0 carrier:0
          colisiones:0 long.colaTX:1000 
          Bytes RX:0 (0.0 B)  TX bytes:0 (0.0 B)
          Interrupci�n:17 

eth2      Link encap:Ethernet  direcci�nHW 00:21:e9:da:3a:b9  
          Direc. inet:192.168.1.111  Difus.:192.168.1.255  M�sc:255.255.255.0
          Direcci�n inet6: fe80::221:e9ff:feda:3ab9/64 Alcance:Enlace
          ACTIVO DIFUSI�N FUNCIONANDO MULTICAST  MTU:1500  M�trica:1
          Paquetes RX:61584 errores:0 perdidos:0 overruns:0 frame:116059
          Paquetes TX:46745 errores:6 perdidos:0 overruns:0 carrier:0
          colisiones:0 long.colaTX:1000 
          Bytes RX:73896966 (73.8 MB)  TX bytes:6057106 (6.0 MB)
          Interrupci�n:16 

eth2:1    Link encap:Ethernet  direcci�nHW 00:21:e9:da:3a:b9  
          Direc. inet:192.168.1.112  Difus.:192.168.1.255  M�sc:255.255.255.0
          ACTIVO DIFUSI�N FUNCIONANDO MULTICAST  MTU:1500  M�trica:1
          Interrupci�n:16 

lo        Link encap:Bucle local  
          Direc. inet:127.0.0.1  M�sc:255.0.0.0
          Direcci�n inet6: ::1/128 Alcance:Anfitri�n
          ACTIVO LOOPBACK FUNCIONANDO  MTU:16436  M�trica:1
          Paquetes RX:108 errores:0 perdidos:0 overruns:0 frame:0
          Paquetes TX:108 errores:0 perdidos:0 overruns:0 carrier:0
          colisiones:0 long.colaTX:0 
          Bytes RX:19345 (19.3 KB)  TX bytes:19345 (19.3 KB)
spenap@mannin:~$ ping -I eth2:1 cymru
PING cymru (192.168.1.141) from 192.168.1.111 eth2:1: 56(84) bytes of data.
64 bytes from Cymru (192.168.1.141): icmp_seq=1 ttl=128 time=1.26 ms
64 bytes from Cymru (192.168.1.141): icmp_seq=2 ttl=128 time=2.06 ms
64 bytes from Cymru (192.168.1.141): icmp_seq=3 ttl=128 time=2.41 ms
64 bytes from Cymru (192.168.1.141): icmp_seq=4 ttl=128 time=2.38 ms
64 bytes from Cymru (192.168.1.141): icmp_seq=5 ttl=128 time=2.07 ms
64 bytes from Cymru (192.168.1.141): icmp_seq=6 ttl=128 time=2.06 ms
64 bytes from Cymru (192.168.1.141): icmp_seq=7 ttl=128 time=4.73 ms
64 bytes from Cymru (192.168.1.141): icmp_seq=8 ttl=128 time=2.04 ms
^C
--- cymru ping statistics ---
8 packets transmitted, 8 received, 0% packet loss, time 7009ms
rtt min/avg/max/mdev = 1.262/2.381/4.733/0.948 ms
\end{verbatim}
}

Como segunda prueba en este ejercicio, levant� una interfaz ``eth0:1'' en un Sheeva plug, con la siguiente configuraci�n:

{\footnotesize
\begin{verbatim}
root@sgarba:~# ifconfig eth0:1 10.10.0.2  broadcast 10.10.0.255 netmask 255.255.255.0
root@sgarba:~# ifconfig eth0:1
eth0:1    Link encap:Ethernet  HWaddr 00:50:43:01:d9:08  
          inet addr:10.10.0.2  Bcast:10.10.0.255  Mask:255.255.255.0
          UP BROADCAST RUNNING MULTICAST  MTU:1500  Metric:1
          Interrupt:11 
root@sgarba:~# netstat -rn
Kernel IP routing table
Destination     Gateway         Genmask         Flags   MSS Window  irtt Iface
10.10.0.0       0.0.0.0         255.255.255.0   U         0 0          0 eth0
192.168.1.0     0.0.0.0         255.255.255.0   U         0 0          0 eth0
0.0.0.0         192.168.1.1     0.0.0.0         UG        0 0          0 eth0
\end{verbatim}
}

Y una interfaz ``br0:1'' en el router, con la siguiente.

{\footnotesize
\begin{verbatim}
# ifconfig br0:1 10.10.0.3 netmask 255.255.255.0 broadcast 10.10.0.255
# ifconfig br0:1
br0:1      Link encap:Ethernet  HWaddr 00:23:69:5C:4C:18  
           inet addr:10.10.0.3  Bcast:10.10.0.255  Mask:255.255.255.0
           UP BROADCAST RUNNING MULTICAST  MTU:1500  Metric:1
# netstat -rn
Kernel IP routing table
Destination     Gateway         Genmask         Flags   MSS Window  irtt Iface
10.10.0.0       0.0.0.0         255.255.255.0   U        40 0          0 br0
192.168.1.0     0.0.0.0         255.255.255.0   U        40 0          0 br0
83.165.0.0      0.0.0.0         255.255.248.0   U        40 0          0 vlan1
127.0.0.0       0.0.0.0         255.0.0.0       U        40 0          0 lo
0.0.0.0         83.165.0.1      0.0.0.0         UG       40 0          0 vlan1
\end{verbatim}
}

A continuaci�n, hice un ping desde el router al sheeva:

{\footnotesize
\begin{verbatim}
# ping 10.10.0.2
PING 10.10.0.2 (10.10.0.2): 56 data bytes
64 bytes from 10.10.0.2: seq=0 ttl=64 time=1.543 ms
64 bytes from 10.10.0.2: seq=1 ttl=64 time=0.727 ms
64 bytes from 10.10.0.2: seq=2 ttl=64 time=0.743 ms
64 bytes from 10.10.0.2: seq=3 ttl=64 time=0.817 ms
64 bytes from 10.10.0.2: seq=4 ttl=64 time=0.736 ms
64 bytes from 10.10.0.2: seq=5 ttl=64 time=0.736 ms

--- 10.10.0.2 ping statistics ---
6 packets transmitted, 6 packets received, 0% packet loss
round-trip min/avg/max = 0.727/0.883/1.543 ms

\end{verbatim}
}

Consultando la tabla ARP, vemos como ahora hay dos entradas con la misma mac, una con la IP que ten�a originalmente ``eth0'' en el Sheeva, y otra con la IP que le asignamos a ``eth0:1''

{\footnotesize
\begin{verbatim}
# arp
Mannin (192.168.1.111) at 00:21:E9:DA:3A:B9 [ether]  on br0
? (10.10.0.2) at 00:50:43:01:D9:08 [ether]  on br0
sgarba (192.168.1.139) at 00:50:43:01:D9:08 [ether]  on br0
Cymru (192.168.1.141) at 00:1C:BF:37:11:92 [ether]  on br0
\end{verbatim}
}

\section{Ejercicio 3}

\textbf{Lista y describe brevemente el contenido del directorio {\it /etc/network}}

{\footnotesize
\begin{verbatim}
spenap@mannin:~$ ls /etc/network
if-down.d  if-post-down.d  if-pre-up.d  if-up.d  interfaces
\end{verbatim}
}

El fichero {\it /etc/network/interfaces} permite definir la configuraci�n de las interfaces.

En el directorio se encuentran cuatro directorios y un archivo. La p�gina de manual de interfaces indica que contienen scripts que se ejecutan cuando se 
han completado los eventos que les dan nombre (al levantar y bajar una interfaz, antes de levantarla y tras bajarla). El fichero {\it network/interfaces}
permite tambi�n fijar comportamientos en las interfaces ante esos eventos. 

\end{document}
