% Definimos el estilo del documento
\documentclass[12pt]{article}
\usepackage[spanish]{babel}

% Acentos y e�es
\usepackage[latin1]{inputenc}
\usepackage{url}
\usepackage{hyperref}

% Definimos titulo, autor, fecha.
\title{Laboratorio II}
\author{Sim�n Pena Placer}
\date{\today}

%Empieza el documento
\begin{document}

% Generamos titulo e indice de contenidos
\maketitle
% \tableofcontents

% Empezamos capitulos
\section{Ejercicio 1}

\textbf{En el laboratorio anterior se efectu� una pregunta relacionada con la MTU de una interfaz de red. Localiza el c�digo de la funci�n eth\_change\_mtu en el c�digo fuente del kernel. L�stalo y expl�calo brevemente. �Es coherente el c�digo con los resultados obtenidos en el laboratorio I?}

La funci�n se encuentra en {\it net/ethernet/eth.c}

{\footnotesize
\begin{verbatim}
/**
 * eth_change_mtu - set new MTU size
 * @dev: network device
 * @new_mtu: new Maximum Transfer Unit
 *
 * Allow changing MTU size. Needs to be overridden for devices
 * supporting jumbo frames.
 */
int eth_change_mtu(struct net_device *dev, int new_mtu)
{
        if (new_mtu < 68 || new_mtu > ETH_DATA_LEN)
                return -EINVAL;
        dev->mtu = new_mtu;
        return 0;
}
EXPORT_SYMBOL(eth_change_mtu);
\end{verbatim}
}

El c�digo recibe una estructura {\it net\_device}, que representa a la interfaz de red, y un mtu nuevo. Comprueba que ese MTU no sea inferior a 68, y no supere
{\it ETH\_DATA\_LEN}, una constante definida en {\it include/linux/if\_ether.h} con valor 1500 (el mismo que ve�amos en el laboratorio anterior usando {\it ifconfig})

{\footnotesize
\begin{verbatim}
#define ETH_DATA_LEN    1500            /* Max. octets in payload        */
\end{verbatim}
}

Si el MTU recibido cumple esas condiciones, se actualiza. En caso contrario, se devuelve un error (argumento inv�lido).

\end{document}
