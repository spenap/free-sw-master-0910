In this report, an analysis about the WebKit Open Source project has been made. 
Most important contributors --all time and last year's, both committers and mailing list writers-- have been identified. Companies behind the development have been confirmed: it was clear that Apple was the main supporter of WebKit, but Google's contribution, while far from Apple's, is also noteworthy. However, Nokia's contribution was not that clear.

Also regarding to companies, it was also made clear that the development is being carried out in an enterprise-mode. Both code development and mailing list handling is being made during the office time, and holiday periods are clear through the project's life. 

About project's health and activity, two facts were presented. The Lorenz Curve and Gini coefficient showed us that development was concentrated on a reduced group of people (typically Apple's developers), while temporal series analysis showed us how the average activity was increasing steadily.

However, interesting research could still be done. The first one is related to Gecko: the origin of this work. Is Gecko increasing its activity at the same rate as WebKit? An analysis of Mozilla's code should be carried on to be able to compare both. 

Another question is related to milestones in the project. While we identified November 2007 as a milestone when analyzing the repository, searching the Web only showed that the WebKit team achieved HTML5 media support, that Android appeared and chose WebKit and that Committer and Reviewer policy changed\footnote{Surfin' Safari - The WebKit Blog - 2007 - November \url{http://webkit.org/blog/date/2007/11/}}

Finally, the analysis could be extended to cover different types of actions in the commits, the Lorenz Curve could be calculated for different groups of developers, and more tests could be done.
