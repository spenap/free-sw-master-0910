\documentclass{beamer}

\usepackage[english]{babel}
\usepackage[latin1]{inputenc}

\usepackage{color}
\usepackage{multimedia}

\usetheme{Copenhagen}
\beamertemplateballitem
\setbeamercovered{transparent}
\setbeamertemplate{navigation symbols}{}

% Declaraci�n de im�genes a usar 
\pgfdeclareimage[width=45px]{cc-license}{images/cc-by-sa.png}
\pgfdeclareimage[width=200px]{browser-share}{images/browser-share.pdf}

\pgfdeclareimage[width=160px]{sloc}{images/sloc.pdf}
\pgfdeclareimage[width=160px]{commits-month}{images/commitsByMonth.pdf}
\pgfdeclareimage[width=160px]{commits-day}{images/commitsByDay.pdf}
\pgfdeclareimage[width=160px]{commits-hour}{images/commitsByHour.pdf}
\pgfdeclareimage[width=160px]{commits-lorenz}{images/lorenz.pdf}

\pgfdeclareimage[width=160px]{messages-month}{images/messagesByMonth.pdf}
\pgfdeclareimage[width=160px]{messages-day}{images/messagesByDay.pdf}
\pgfdeclareimage[width=160px]{messages-hour}{images/messagesByHour.pdf}

\pgfdeclareimage[width=160px]{bugs-month}{images/unassignedBugsByMonth.pdf}

% T�tulo, subt�tulo, autor, afiliaci�n y fecha
\title{Webkit Project -- A FLOSS report}
\subtitle{Dynamics of Libre Software Communities}
\author{Sim�n Pena Placer}
\institute{M�ster Software Libre, 2009-2010. A Coru�a Edition}
\date{\today}

\begin{document}
\nocite{webkit:home}

% Portada
\begin{frame}
	\titlepage
\end{frame}

\begin{frame}
	\frametitle{License}
	\begin{center}
	\pgfuseimage{cc-license}
	\end{center}
This work is licensed under the Creative Commons Attribution-Share Alike 3.0 Spain License. To view a copy of this license, visit \url{http://creativecommons.org/licenses/by-sa/3.0/es/} or send a letter to Creative Commons, 171 Second Street, Suite 300, San Francisco, California, 94105, USA.
\end{frame}

% �ndice
\begin{frame}
	% Con este comando se pone el t�tulo de la diapositiva
	\frametitle{Content}
	\tableofcontents
\end{frame}

% Con esto hacemos que al entrar en cada secci�n aparezca el �ndice
% con la secci�n actual remarcada
% Tambi�n es posible la opci�n \AtBeginSubSection
\AtBeginSubsection[]
{
  \begin{frame}<beamer>
    \frametitle{Outline}
    \tableofcontents[current,currentsubsection]
  \end{frame}
}

\section{Introduction}
\begin{frame}{Introduction (I)}{About this work}
\begin{itemize}
	\item<1-> An Open Source Project will be analyzed
	\item<2-> Which one? \uncover<4->{Why?}
	\item<3-> WebKit project was chosen
	\item<5-> There is a technology war between web browsers
	\begin{itemize}
		\item<6-> Proprietary browser Internet Explorer rules the market
		\item<7-> However, {\it times may be changing}: Open Source browsers share is growing
	\end{itemize}
\end{itemize}
\end{frame}

\begin{frame}{Introduction (II)}{The Browser War}
	\begin{center}
	\pgfuseimage{browser-share}
	\end{center}	
\end{frame}

\begin{frame}{Introduction (II)}{Beyond the browser war}
\begin{itemize}
	\item The war has been extended to applications using web browser's engines.
	\item Gecko, Mozilla's engine, is one of the most used.
	\item WebKit, Apple's Safari engine, is experimenting an enormous growth
\end{itemize}
\end{frame}

\begin{frame}{Introduction (III)}{About this work}
\begin{itemize}
	\item {\it Which engine is better suited to your application development?}
	\item This work will help you decide, by analyzing the WebKit project. 
	\item A compared analysis with Gecko is out of the scope of this work
	\begin{itemize}
		\item Gecko's code is too coupled with Mozilla's
		\item It is hard to analyze it alone 
	\end{itemize} 
\end{itemize}
\end{frame}

\begin{frame}{Introduction (IV)}{Some info about the WebKit project}
\begin{itemize}
	\item WebKit is Apple's Safari layout engine
	\item It started from KDE's KHTML and KJS engines 
	\item Its development has been opening more and more
	\item Nowadays its code is under the LGPL or Apache Licenses 
\end{itemize}
\end{frame}

\begin{frame}{Introduction (and V)}{Projects using WebKit}
There are many projects using WebKit. Regarding to Web browsers:
\begin{itemize}
	\item Safari and Safari Mobile, from Apple's OS X and iPhone OS, respectively
	\item Google Chrome Browser and Chromium Open Source project
	\item Android's Web browser
	\item Nokia's S60 browser
	\item Epiphany, who has moved from Gecko
	\item Others\dots
\end{itemize}
\end{frame}

\section{Methodology and tools}

\begin{frame}{Methodology}
The following project resources will be analyzed
\begin{itemize}
	\item Source code repository, located at \url{http://svn.webkit.org/repository/webkit/trunk/} 
	\item Developers mailing list, located at \url{https://lists.webkit.org/pipermail/webkit-dev/}
	\item Unassigned bugs mailing list, located at \url{https://lists.webkit.org/pipermail/webkit-unassigned/}
\end{itemize}
\end{frame}

\begin{frame}{Tools}
\begin{itemize}
	\item Libresoft tools\cite{libresoft:tools} will be used
	\begin{itemize}	
		\item cvsanaly: \url{http://git.libresoft.es/cvsanaly/}
		\item mlstats: \url{http://forge.morfeo-project.org/plugins/scmsvn/viewcvs.php/mailingliststat/?root=libresoft-tools}
		\item guilty: \url{http://git.libresoft.es/guilty/}
	\end{itemize}
	\item Project data will be exported into a database via these tools
	\item A series of queries will be used in order to gather relevant information
	\item R\cite{r:citation} will be used to issue the queries and export their results
	\item Some queries from Flossreport were also used 
\end{itemize}
\end{frame}

\begin{frame}{Source Code Repository}
\begin{itemize}
	\item Most important info will be summarized: number of commits and committers, lines of code, etc.
	\item Most active committers will be identified, both all-time and last year
	\item Companies behind development will be identified, and its relevance evaluated
	\item Information about trends will be extracted, different periods will be shown
\end{itemize}
\end{frame}

\begin{frame}{Developers Mailing List}
\begin{itemize}
	\item Most important info will be summarized: number of emails sent and writers, lifespan, etc.
	\item Most active posters will be identified, both all-time and last year
	\item Companies behind development will be identified, and its relevance evaluated
	\item Information about trends will be extracted, different periods will be shown.
\end{itemize}
\end{frame}

\begin{frame}{Unassigned bugs mailing list}
\begin{itemize}
	\item All-time number of emails received will be shown
	\item Mailing list trends will be exposed
\end{itemize}
\end{frame}
	
\section{Results}

\subsection[Repository]{Source code repository}

\begin{frame}{Summary}
\begin{table}[ht]
	\begin{center}
	\begin{tabular}{lr}
	  \hline
	 Concept & Count \\ 
	  \hline
	Number of commits & 44143 \\ 
	Number of committers & 195 \\
	Number of files under version control & 80860\\
	Number of lines & 824955\\
	Years of activity & 9\\
	   \hline
	\end{tabular}
	\caption{Brief summary of the repository's activity}
	\label{repo:summary}
	\end{center}
\end{table}
\end{frame}

\begin{frame}{Programming language distribution}
\begin{figure}[!hptb]
\pgfuseimage{sloc}
\caption{Programming language distribution}
\label{sloccount:summary}
\end{figure}
\end{frame}

\begin{frame}{All-time top 10 committers}
% latex table generated in R 2.9.2 by xtable 1.5-6 package
% Wed Jan 20 20:01:11 2010
\begin{table}[!htpb]
\begin{center}
\begin{tabular}{rlr}
  \hline
 & Committer & Commit count \\ 
  \hline
1 & darin & 3583 \\ 
  2 & hyatt & 2158 \\ 
  3 & eric@webkit.org & 1967 \\ 
  4 & mjs & 1620 \\ 
  5 & hausmann@webkit.org & 1174 \\ 
  6 & rjw & 1104 \\ 
  7 & darin@apple.com & 1076 \\ 
  8 & kocienda & 958 \\ 
  9 & mitz@apple.com & 945 \\ 
  10 & mrowe@apple.com & 941 \\ 
   \hline
\end{tabular}
\caption{Top 10 committers. Multiple accounts ignored}
\label{commits:top20}
\end{center}
\end{table}

\end{frame}

\begin{frame}{All-time top 10 committers -- accounts grouped}
% latex table generated in R 2.9.2 by xtable 1.5-6 package
% Wed Jan 20 20:01:11 2010
\begin{table}[!htpb]
\begin{center}
\begin{tabular}{rlr}
  \hline
 & Committer & Commit count \\ 
  \hline
1 & darin & 4876 \\ 
  2 & hyatt & 3042 \\ 
  3 & eric & 1967 \\ 
  4 & mjs & 1842 \\ 
  5 & hausmann & 1417 \\ 
  6 & andersca & 1376 \\ 
  7 & ggaren & 1287 \\ 
  8 & weinig & 1263 \\ 
  9 & ap & 1252 \\ 
  10 & aroben & 1164 \\ 
  11 & rjw & 1104 \\ 
  12 & mitz & 968 \\ 
  13 & kocienda & 958 \\ 
  14 & mrowe & 941 \\ 
  15 & oliver & 900 \\ 
  16 & cblu & 885 \\ 
  17 & eseidel & 860 \\ 
  18 & beidson & 759 \\ 
  19 & sullivan & 724 \\ 
  20 & adele & 695 \\ 
   \hline
\end{tabular}
\caption{Top 20 committers. Multiple accounts grouped}
\label{commits:top20grouped}
\end{center}
\end{table}

\end{frame}

\begin{frame}{2009 top 10 committers}
% latex table generated in R 2.9.2 by xtable 1.5-6 package
% Wed Jan 20 20:01:11 2010
\begin{table}[!htpb]
\begin{center}
\begin{tabular}{rlr}
  \hline
 & Committer & Commit count \\ 
  \hline
1 & eric@webkit.org & 1646 \\ 
  2 & abarth@webkit.org & 474 \\ 
  3 & hausmann@webkit.org & 440 \\ 
  4 & kov@webkit.org & 387 \\ 
  5 & darin@apple.com & 385 \\ 
  6 & mrowe@apple.com & 364 \\ 
  7 & simon.fraser@apple.com & 357 \\ 
  8 & mitz@apple.com & 334 \\ 
  9 & hyatt@apple.com & 322 \\ 
  10 & oliver@apple.com & 320 \\ 
   \hline
\end{tabular}
\caption{Top 10 committers during 2009}
\label{commits:2009top20}
\end{center}
\end{table}

\end{frame}

\begin{frame}{Commits by company}
% latex table generated in R 2.9.2 by xtable 1.5-6 package
% Wed Jan 20 20:01:11 2010
\begin{table}[!htpb]
\begin{center}
\begin{tabular}{rlllr}
  \hline
 & \tiny{Company} & \tiny{First contribution} & \tiny{Last contribution} & \tiny{Commit count} \\ 
  \hline
1 & apple.com & 2007 & 2010 & 12005 \\ 
  2 & webkit.org & 2007 & 2010 & 8753 \\ 
  3 & chromium.org & 2008 & 2009 & 1831 \\ 
  4 & nokia.com & 2009 & 2010 &  63 \\ 
  5 & google.com & 2009 & 2009 &  30 \\ 
  6 & torchmobile.com & 2009 & 2009 &  10 \\ 
  7 & forwardbias.in & 2009 & 2009 &   8 \\ 
   \hline
\end{tabular}
\caption{Number of commits by the company an user is affiliated to}
\label{commits:company}
\end{center}
\end{table}

\end{frame}

\begin{frame}{Yearly activity}
\begin{figure}[!hptb]
\pgfuseimage{commits-month}
\caption{Evolution of yearly activity in the repository}
\label{commits:evo:monthly}
\end{figure}
\end{frame}

\begin{frame}{Weekly activity}
\begin{figure}[!hptb]
\pgfuseimage{commits-day}
\caption{Evolution of weekly activity in the repository}
\label{commits:evo:weekly}
\end{figure}
\end{frame}

\begin{frame}{Daily activity}
\begin{figure}[!hptb]
\pgfuseimage{commits-hour}
\caption{Evolution of daily activity in the repository}
\label{commits:evo:daily}
\end{figure}
\end{frame}

\begin{frame}{Gini coefficient and Lorenz curve}
\begin{figure}[!hptb]
\pgfuseimage{commits-lorenz}
\caption{Lorenz curve for the repository. Gini coefficient is $0.706$}
\label{commits:lorenz}
\end{figure}
\end{frame}

\subsection[Mailing list]{Developers mailing list}

\begin{frame}{Summary}
\begin{table}[ht]
	\begin{center}
	\begin{tabular}{lr}
	  \hline
	Concept & Count \\ 
	  \hline
	Emails sent to the list & 9760 \\ 
	Unique email addresses writing to the list & 1199 \\ 
	Different user names writing to the list & 1132 \\
	Years of activity & 5\\
	   \hline
	\end{tabular}
	\caption{Brief summary of the developers' mailing list}
	\label{dev_mls:summary}
	\end{center}
\end{table}
\end{frame}

\begin{frame}{All-time top 10 posters}
% latex table generated in R 2.9.2 by xtable 1.5-6 package
% Wed Jan 20 20:01:11 2010
\begin{table}[!htpb]
\begin{center}
\begin{tabular}{rlr}
  \hline
 & Username & Email count \\ 
  \hline
1 & darin & 623 \\ 
  2 & mjs & 605 \\ 
  3 & ddkilzer & 307 \\ 
  4 & mrowe & 209 \\ 
  5 & aroben & 206 \\ 
  6 & mike.emmel & 173 \\ 
  7 & eric & 170 \\ 
  8 & hyatt & 165 \\ 
  9 & ggaren & 160 \\ 
  10 & abarth & 147 \\ 
  11 & pkasting & 125 \\ 
  12 & bfulgham & 115 \\ 
  13 & ap & 106 \\ 
  14 & oliver & 102 \\ 
  15 & jorlow &  98 \\ 
  16 & zecke &  91 \\ 
  17 & kevino &  83 \\ 
  18 & jackwootton &  80 \\ 
  19 & jhaygood &  73 \\ 
  20 & vniles &  70 \\ 
   \hline
\end{tabular}
\caption{Top 20 posters}
\label{emails:top20}
\end{center}
\end{table}

\end{frame}

\begin{frame}{2009 top 10 posters}
% latex table generated in R 2.9.2 by xtable 1.5-6 package
% Wed Jan 20 20:01:11 2010
\begin{table}[!htpb]
\begin{center}
\begin{tabular}{rlr}
  \hline
 & Username & Email count \\ 
  \hline
1 & darin & 288 \\ 
  2 & mjs & 233 \\ 
  3 & abarth & 132 \\ 
  4 & eric & 123 \\ 
  5 & pkasting & 106 \\ 
  6 & jorlow &  96 \\ 
  7 & ddkilzer &  86 \\ 
  8 & mrowe &  78 \\ 
  9 & ggaren &  72 \\ 
  10 & aroben &  69 \\ 
   \hline
\end{tabular}
\caption{Top 20 posters during 2009}
\label{emails:2009top20}
\end{center}
\end{table}

\end{frame}

\begin{frame}{Messages by company}
% latex table generated in R 2.9.2 by xtable 1.5-6 package
% Wed Jan 20 20:01:12 2010
\begin{table}[!htpb]
\begin{center}
\begin{tabular}{rlr}
  \hline
 & Domain name & Email count \\ 
  \hline
1 & apple.com & 2329 \\ 
  2 & gmail.com & 2244 \\ 
  3 & webkit.org & 665 \\ 
  4 & chromium.org & 423 \\ 
  5 & google.com & 337 \\ 
  6 & yahoo.com & 221 \\ 
  7 & mac.com & 169 \\ 
  8 & kde.org & 134 \\ 
  9 & kilzer.net & 110 \\ 
  10 & selfish.org &  91 \\ 
   \hline
\end{tabular}
\caption{Number of emails sent by company employees}
\label{emails:company}
\end{center}
\end{table}

\end{frame}

\begin{frame}{Yearly activity}
\begin{figure}[!hptb]
\pgfuseimage{messages-month}
\caption{Evolution of yearly activity in the developers' mailing list}
\label{mails:evo:monthly}
\end{figure}
\end{frame}

\begin{frame}{Weekly activity}
\begin{figure}[!hptb]
\pgfuseimage{messages-day}
\caption{Evolution of weekly activity in the developers' mailing list}
\label{mails:evo:weekly}
\end{figure}
\end{frame}

\begin{frame}{Daily activity}
\begin{figure}[!hptb]
\pgfuseimage{messages-hour}
\caption{Evolution of daily activity in the developers' mailing list}
\label{mails:evo:daily}
\end{figure}
\end{frame}

\subsection[Bug tracking system]{Unassigned bugs mailing list}

\begin{frame}{Summary}
\begin{block}{}
\begin{itemize}
	\item Each time a bug is created, a message is sent to the list
	\item Each time a bug changes it state, another message is sent
	\item 160342 Emails sent to the list
\end{itemize}
\end{block}
\end{frame}

\begin{frame}{Yearly activity}
\begin{figure}[!hptb]
\pgfuseimage{bugs-month}
\caption{Evolution of the monthly activity on the unassigned bugs mailing list}
\label{bugs:evo:monthly}
\end{figure}
\end{frame}

\section{Conclusions}

\begin{frame}{Conclusions}{Some work has been done...}
An analysis about WebKit project has been performed:
\begin{itemize}
	\item The overall trend for the project's growth is clearly positive
	\item Most important contributors have been identified for both the repository and mailing list
	\item The development model has been exposed
	\item Companies supporting the development have been identified
\end{itemize}
\end{frame}

\begin{frame}{Conclusions (and II)}{...but there is still work left to do}
\begin{itemize}
	\item Handling WebKit project as a whole makes it harder to understand

	\begin{itemize}
		\item Analysis performed on smaller parts would be interesting	 
	\end{itemize}

	\item Different queries could show different information currently omitted
	\item Other layouts should be taken into account for the analysis to be completed
	\begin{itemize}
		\item Specific tests should be designed in order to help comparing engines
		\item Gecko engine would be the first candidate	 
	\end{itemize}
\end{itemize}
\end{frame}
	
\bibliographystyle{plain}
\bibliography{webkit-beamer.bib}

% Portada
\begin{frame}
	\titlepage
\end{frame}
\end{document}

